\chapter{Introduction}
\lhead{\emph{Introduction}}
Robots have been around us ever since the time of Leonardo da Vinci. The development in this field has seen a tremendous growth and intelligent robots are emerging in almost every field like automotive, medicine, defence which are just to name a few. We are in an era where intelligence has evolved from Shakey, the world's first intelligent robot to Sophia, the world's first robot who was awarded a citizenship.

\par Huge amount of research have been invested in robots to assist humans in tasks that are monotonous or dangerous. For example, the entry of autonomous machines such as Roomba\cite{1} not only revolutionized the industry of floor cleaning but also has brought robots much closer to humans which were instead confined in spaces around computer scientists. A bigger example is the recent bloom of autonomous cars. Almost all automotive as well as software companies have their dedicated R\&D for self-driving cars and numerous technologies are being developed and tested every day. These autonomous robots not only make our lives easier but also play a crucial role in applications of autonomous exploration such as search and rescue, mine detection, etc. 

\par Autonomous robots are intelligent machines capable of performing tasks in the world by themselves, without explicit human control. These intelligent machines must posses and exhibit the following qualities\cite{6}
\begin{itemize}
    \item Avoid harmful situations especially to humans, itself and fellow robots, unless its designed to do so. 
    \item Work for an extended period of time without any human intervention.
    \item Gain information about the environment by sensing it.
\end{itemize}
This work is about autonomous exploration of a-priori unknown environments with mobile robots. Hence forth all discussions are focused on autonomous mobile robots unless specified.

\section{Autonomous robots}
While many robots can navigate using maps, few can build their own maps. Mobile robots need a map to effectively navigate in their environment. The ability of mobile robots to autonomously move in an unknown environment to gather the sensory information required to build a map for navigation is called autonomous exploration. A good exploration strategy is one that generates a complete or nearly complete map in a reasonable amount of time. Exploration and map-building are extensively researched by robotics community due to its wide range of real-world applications. Such applications may include search and rescue, hazardous material handling, military actions, planetary exploration, path planning, and devastated area exploration\cite{17}. Generally, autonomous robot is able to incrementally construct a model (map) for its environment based on the sensory information gathered in an online fashion, i.e. while navigating through the environment. This process requires choosing the best next location for the robot to visit, planning the shortest path to reach that location and finally controlling the robot’s motion in its journey to that location. 
\par Autonomous mobile robots interact with their environment to get an internal representation of the environment. For any autonomous mobile robot to explore or cover an unknown environment, the following are the prerequisites, 
\begin{itemize}
    \item Accurately map the environment and simultaneously get a position estimate of itself in the mapped environment.
    \item Navigate itself through environment which might be dynamic as well.
    \item Increase its extent of the partially explored map and completely explore the environment.
\end{itemize}

\par Simultaneous localization and mapping technique is often used to construct a map for the environment and localize the robots in it. As the robots move to unexplored new areas, these areas are then included in the map. But there are a number of questions to be answered in designing an autonomous exploration strategy. The key challenges in autonomous exploration are

\begin{itemize}
    \item How does the robot extend its partially explored map?
    \item How a robot plans in order to visit the remaining unexplored areas while minimizing the total traveled distance?
    \item In case of multiple robots, how do they distribute the unexplored space efficiently?
\end{itemize}

\section{Motivation}
If a robot is equipped with a vision sensor, where should it move after it is switched on? In this work we answer this question by using a frontier based exploration approach wherein the bot attempts to look at, thus map every boundary which separates the reachable space from the uncharted, known as the frontiers. Ideally the bot has to move to these boundaries to expand its map. Given, the robot explores the frontiers it senses, which frontier is to be selected as its next position in the environment such that the exploration is efficient? Using Wi-Fi how is the robot motion planned such that it covers the entire environment? How do multiple such robots distribute the environment and efficiently explore with minimum communication possible? All the above questions posed form the key motivation for Wi-Coverage. Through this work, we would like to introduce Wi-Fi as a readily available ubiquitous sensor to help assist explorations or perform search operations in huge places like airports, mall, etc.
\par An additional motivation for Wi-Coverage are Robot operating system (ROS) and Gazebo. ROS is an open source software framework available which makes a roboticist's life better by providing low level drivers and interface with physical parts of the robot. Gazebo developed by Willowgarage is a simulation platform capable of simulating life like real world scenarios much better than the older platforms like Stage. Several coverage algorithms are devised using ROS, but only few of them are directed towards online mapping and multi robot explorations, and on the idea of simulating using Gazebo. 

\section{Frontier Exploration}
Given the ability to sense the environment around it, when a mobile robot is switched on, how does it know where to go next which increases the extent of its map? Well, a simple approach would be to make the robot move in a pattern, for example in concentric circles\cite{9}. Though this approach would extend the horizons of the map, it would not guarantee that we completely explore the environment. A much clever method would be to reach an empty space in the partially explored environment. For example, imagine a scenario where we enter a new apartment. In an attempt to explore the new environment, we move to the boundaries up to where we are able to see and are reachable. We finally explore the whole place like rooms and porch by repeating the same until there is no such boundary left.  Exploration can be described as the act of moving through an unknown environment while building a map that can be used for subsequent navigation\cite{10}. This approach is popularly known as the frontier based exploration which we used in this work.

\par Exploring frontiers will eventually complete visiting every place in the environment. The way in which the frontiers are explored, decides the effectiveness of the algorithm. In other words, which frontier to pick next such that the robot completes the entire space in the least amount of time? This is our next challenge. A popular solution is to maximize the information gain at every frontier such that the new information about the environment obtained at every position the robot visits is maximum in the current map. Though this approach is efficient, the bots path is highly unpredictable and depends on the sensors precision and sensitivity. In this work we propose a variant to this approach by using Wi-Fi sensing i.e sensing the signal strength from various Wi-Fi routers to aid the bot in exploring the environment in a more systematic way. In addition to using the information gain formulation ,we associate the RSSI from the surrounding Wi-Fi routers to it. This helps the bot cluster the environment and explore it systematically. This approach does not require any prior information about the environment i.e all processes are performed online. This novel approach particularly has an edge over traditional methods in terms of communication between robots in collaborative robotics.

\section{Wi-Fi as an additional sensor} 

Very often, it is evident that using multiple sensors outperform the efficiency, when instead used a single sensor. Additional information is either fused with the primary information or directly used in consequent stages to increase accuracy and performance of the policy. In some cases, augmenting with additional data can help overcome the limitations of the actual sensor. Wi-Fi routers have become ubiquitous for quick internet access in most urban settings including offices, homes, and public spaces such as malls and airports. Most robots are typically equipped with a Wi-Fi radio for communication as well. Our conjecture was that we could use Wi-Fi as an additional sensing modality which can provide improved solutions when augmented with the primary sensor(s) information. In this work, we demonstrate how the policy of picking frontiers can be improved in case of frontier exploration such that the robots path is systematic and predictable. We show that by using this additional sensor we can overcome the range limitation of a stereo camera and understand the environment in a much greater sense.

% \section{goals}

% % Two of the significant branches in the robotics are autonomous robot exploration and cooperative robotics. Autonomous robot exploration involves robot exploring a space without any human interaction or assistance. They are differentiated from remote controlled, or tele-operated mobile robots in the fact that there is no human in the loop controlling a robots movement. In other words, the robots decision and executes of its choices are all computerized\cite{25}.
% The areas of robotic research in the second half of twentieth century have covered a breadth of topics including human-assisting robots, home automation robots, industrial manufacturing robots, search and rescue robots and many more. Two of the significant branches in robotics are autonomous robot exploration and cooperative robotics to which we would like to contribute. There are two main goals in this thesis. The first one is to use vision and inexpensive off the shelf components and implement a frontier based exploration available in its best form which is by maximizing information gain. The other one is to develop an online policy for exploring an unknown environment using commodity WiFi which extremely minimizes the communication efforts between robots in a cooperative environment while almost maintaining the maximum coverage and time possible by efficient exploration techniques like frontier based exploration using information gain. As a part of this we will see on how to use commodity Wifi as an additional sensor which is very much an abundant and useful resource found in every indoor space, to aid the main sensors solving the coverage problem in a much more systematic way.  

\section{Thesis organization}    
We begin, in Chapter 2, with an overview of research done in the field of exploration and numerous coverage strategies. Chapter 3 describe our implementation of frontier based exploration using a vision sensor and ROS. We then describe our novel coverage strategy, Wi-Coverage in both single and multi robot scenarios. The pipeline and design of frontier selection using Wi-Coverage is described in detail. Later in the chapter, we present some engineering issues faced in implementing the frontier based exploration and methods to solve them. 
A robot is a conglomeration of complex software and hardware components. Chapter 3 also describes the robot, Darwin used in this work and at DRONES Lab. We present the software ROS: its essential components and protocols, which we used in this work and its high level packages like RTAB-Map and move base. Experiments and results are presented in Chapter 5. Chapter 6 contains concluding remarks and an outline of future work. 

\section{Contributions}
The main contributions of this thesis are as follows:
\begin{itemize}
    \item An implementation of frontier based exploration using vision and solving the engineering problems faced while realizing it in the real world.
    \item A coverage strategy which uses a freely available resource, Wi-Fi, to assist the robot(s) in exploration of unknown environment. This strategy also considerably minimizes the communication required between robots in a multi robot scenario.
    \item Design of simulation environments with software configurable Wi-Fi routers with multiple signal strength models in Gazebo simulator.
\end{itemize}