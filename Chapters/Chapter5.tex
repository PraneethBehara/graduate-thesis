\chapter{Conclusions}
\lhead{\emph{Conclusions}}
This chapter presents the highlights of the material covered in the preceding chapters, going through the main contributions and discuses some avenues of future work.

\section{Summary of contributions}
We began with the observation that Wi-Fi information can be used in assisting autonomous robots in exploration. Chapter 2 presented a summary of related technical aspects which laid the foundation for the following chapters. It introduced autonomous robots and their prerequisites to become autonomous explorers. Then we looked into various forms of mapping techniques to understand the environment in a robots perspective. Coverage problem, being one of the motivations for this work is discussed in detail. A summary of existing exploration methods and coverage strategies for both single and multi robot scenarios were then presented.
\par Chapter 3 which is the core of this work introduces a novel approach to handle coverage in indoor spaces. Wi-Coverage, the coverage strategy which integrates Wi-Fi sensing with frontier based exploration to systematically and completely cover any given environment has been discussed. This chapter also presented the robot Darwin used in this work. In particular the hardware and software construction and components used were outlined.
\par Chapter 4 highlights the experiments and results conducted and evaluated using Wi-Coverage. Frontier exploration using optimal gain approach is a gold standard in autonomous exploration. We compare Wi-Coverage with it in various scenarios and environments. First, we created two different maps using Gazebo to simulate real world scenarios for simulations. Second, in order to simulate Wi-Fi, we created our own nodes in ROS. We also gave APs the ability to simulate two different signal strength models which helps in extensive testing of Wi-Coverage. Since exploration is time sensitive, we choose full coverage time in different signal propagation environments and percentage area covered vs time as our metrics. In order to truly appreciate the novelty of Wi-Coverage, robot trajectories are plotted and described. 
\par We show that our approach uses a freely available sensor and uses it to assist exploration without adding any cost or bulk to the robot. Finally, we prove that Wi-Coverage not only has a similar coverage time as the gold standard but also systematically covers the space allowing some predictability of the robots movement. It clearly shows that using Wi-Coverage the robot can overcome the limitations of its primary sensors and can understand its environment, plan its motion in a much greater sense. Also, we satisfy all the requirements for an algorithm to be reliable and realizable in the real world.
\par Lastly, we present some engineering issues faced in this work in order to make frontier exploration realizable in the real world using a Turtlebot, a stereo camera and ROS. We also provide solutions, both traditional and custom created for this work.

\section{Future work}
There is plenty of scope for future improvements and improvisations to the work presented in this thesis. Some fulfillment's and interesting directions include:

\textbf{Real time experiments} : The results obtained for frontier based exploration approach using a stereo camera and required packages from ROS have been extensively tested on a physical robot and in a real environment. But for the Wi-Coverage results are obtained using simulated environments in gazebo simulator. Though the simulations are close enough to reality, these have ideal conditions which suggests a future enhancement of testing the novel algorithm in real environments and with multiple physical robots. Also this algorithm can be tested using various sensors and results can be analyzed. 

\textbf{Exploration in 3D} : Wi-Coverage has been designed and implemented on a derived 2D occupancy grid. The robot senses the signal strength from ground level and it is assumed to be constant along the \textit{z} axis wrt the robot, i.e the altitude. An enhancement to this approach is to use 3D occupancy grid where each cell is now a voxel and area is now volume. Identifying frontiers are much challenging and computationally less expensive techniques can be developed. Clustering in layers of a 3D space using Wi-Fi signal strength and giving robots(ex aerial UAVs or robots which walk the stairs) a sense of altitude to explore and map large structures such as airports will be our future work. 

\textbf{Localize Wi-Fi routers} : Many applications such as positioning Wi-Fi routers for best coverage in a given space, indoor localization, etc requires the position estimate of the Wi-Fi routers. As Wi-Coverage involves listening to APs for signal strengths, the database collected while exploring can be used to localize them. Methods like \cite{29}\cite{30} can be used for AP localization and thereby use the routers as landmarks to improve the robot localization. 

\section{Concluding Remark}
First, this thesis has presented a way of implementing frontier based exploration with a mobile robot using just a vision sensor and extensively discussed the engineering problems faced. Second, we developed a novel online coverage solution for mobile robots in operating spaces using Wi-Fi as an additional sensor. We hope that our work, and specifically the conjecture of using Wi-Fi as an additional sensor will be useful in the development of fully autonomous exploration systems and for improving various other applications.